%! Author = fivv
%! Date = 2023/11/23


% Preamble
\documentclass{article}

% Packages
\usepackage[UTF8]{ctex}
\usepackage[T1]{fontenc}
\usepackage{hyperref}

%\setmainfont{Times New Roman}
\setCJKmainfont{SimSun}[BoldFont=SimHei, ItalicFont=KaiTi]
\setCJKsansfont{SimHei}
\setCJKmonofont{SimSun}


% Document
\begin{document}

    % raw text
    \begin{verbatim}
    ==>> 原样输出
    \documentclass{...} % 文档类声明
    \usepackage{...} % 宏包声明
    \begin{document}
    % 文档内容
    \end{document}
    % 此后的内容不会被编译
    \end{verbatim}
%    ==>> input \\
%    \begin{document}
    include\_test.tex
\end{document}

%    ==>> include  \\
%    \begin{document}
    include\_test.tex
\end{document}
    ==>> description \\
    \begin{description}
        \item[引擎] 全称为排版引擎,是编译源代码并生成文档的程序,如 \hologo{pdfTeX}、\hologo{XeTeX} 等。有时也称为编译器。
        \item[格式] 是定义了一组命令的代码集。\LaTeX{} 就是最广泛应用的一个格式,高德纳本人还编写了一个简单的 \hologo{plainTeX} 格式,
        没有定义诸如 \cmd{document\-class} 和 \cmd{section} 等等命令。
    \end{description}

    ==>> space and paragraph \\
    Several spaces equal one space.Front spaces are ignored.

    An empty line starts a new paragraph.
    A \verb|\par| command also starts a new paragraph.

    ==>> word space \\
    Fig.1 Fig.~1\\

    ==>> new line \newline
    使用 \verb|\newline| 断行的效果
    \newline
    与使用 \verb|\linebreak| 断行的效果
    \linebreak
    进行对比。

    ==>> 断词 \newline
    I think this is: su\-per\-cal\-%
    i\-frag\-i\-lis\-tic\-ex\-pi\-%
    al\-i\-do\-cious. \\
    aaaaaaaaaaaaaaaaaaaaaaaaaaaaaaaaaaaaaaaaaa\-Aaaaaaaaaaaaaaaa\-Aaaaaaaaaaaaaaaaa\-Aaaaaaaaaaaaaaaaaaaaaaaaaaaaaaaaaaaaaaa \\
    \newpage

    ==>> 目录 \newline
    \tableofcontents
    \newpage

    ==>> 章节 \newline


    \section{section}\label{sec:section}

    \subsection{subsection}\label{subsec:subsection}

    \section*{section*}\label{sec:section*}

    \section*{section*2}\label{sec:section*2} \addcontentsline{toc}{section}{section*2}

%    \chapter{chapter}\label{ch:chapter}

    \newpage

%    ==>> 标题页 \\
    \title{Test title}
    \author{ Mary\thanks{E-mail:*****@***.com}
    \and Ted\thanks{Corresponding author}
    \and Louis}
    \date{\today}
    \maketitle


    \section{交叉引用}\label{sec:crossref}
    ==>> 交叉引用 \\
%    \begin{example}
    A reference to this subsection
    \label{sec:this} looks like:
    ``see section~\ref{sec:this} on
    page~\pageref{sec:this}.''
%    \end{example}

    ==>> 脚注 \\
    “天地玄黄,宇宙洪荒。日月盈昃,辰宿列张。”\footnote{出自《千字文》。}

    ==>> 边注 \\
    \marginpar{\footnotesize 边注较窄,不要写过多文字,最好设置较小的字号。}


    ==>> 列表 \\
    \begin{itemize}
        \item[*] first
        \item second
        \item third
        \item[+] fourth
    \end{itemize}

    \begin{enumerate}
        \item first
        \item second
        \item third
        \item[+] fourth
        \item fifth
    \end{enumerate}

    \renewcommand{\labelitemi}{\ddag}
    \renewcommand{\labelitemii}{\dag}
    \begin{itemize}
        \item first
        \begin{itemize}
            \item second
            \item third
        \end{itemize}
        \item fourth
    \end{itemize}


    ==>> 对齐 \\
    \begin{center}
        Centered text using a
        \verb|center| environment.
    \end{center}

    \begin{flushleft}
        Left-aligned text using a
        \verb|flushleft| environment.
    \end{flushleft}

    \begin{flushright}
        Right-aligned text using a
        \verb|flushright| environment.
    \end{flushright}

    \centering
    Centered text using a
    \verb|\centering| command.

    \raggedright
    Left-aligned text using a
    \verb|\raggedright| command.

    \raggedleft
    Right-aligned text using a
    \verb|\raggedleft| command.

    \newline

    \begin{tabular}{|c|}
        center-\\aligned \\
    \end{tabular}

    \begin{tabular}{|l|}
        left-\\aligned \\
    \end{tabular}

    \begin{tabular}{|r|}
        right-\\aligned \\
    \end{tabular}


\end{document}

